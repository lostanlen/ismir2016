% -----------------------------------------------
% Template for ISMIR Papers
% 2015 version, based on previous ISMIR templates
% -----------------------------------------------

\documentclass{article}
\usepackage{ismir,amsmath,cite}
\usepackage{graphicx}
\usepackage{color}
\usepackage{mathrsfs}
\usepackage[english]{babel}
\usepackage{caption}
\usepackage{subfig, color}
\usepackage{microtype}
\usepackage{IEEEtrantools}
\sloppy

\usepackage{xspace}
\newcommand*{\eg}{e.g.\@\xspace}
\newcommand*{\ie}{i.e.\@\xspace}
\newcommand*{\etal}{et al.\@\xspace}

% Title.
% ------
\title{Learning pitch invariants for instrument recognition}


% Single address
% To use with only one author or several with the same address
% ---------------
\oneauthor
{Names should be omitted for double-blind reviewing}
{Affiliations should be omitted for double-blind reviewing}
%\oneauthor
%{Vincent Lostanlen, Carmine Emanuele Cella, and St\'{e}phane Mallat}
%{\'{E}cole normale sup\'{e}rieure}

% Two addresses
% --------------
%\twoauthors
%  {First author} {School \\ Department}
%  {Second author} {Company \\ Address}

% Three addresses
% --------------
%\threeauthors
 %{First author} {Affiliation1 \\ {\tt author1@ismir.edu}}
 %{Second author} {\bf Retain these fake authors in\\\bf submission to preserve the formatting}
 %{Third author} {Affiliation3 \\ {\tt author3@ismir.edu}}

% Four addresses
% --------------
%\fourauthors
%  {First author} {Affiliation1 \\ {\tt author1@ismir.edu}}
%  {Second author}{Affiliation2 \\ {\tt author2@ismir.edu}}
%  {Third author} {Affiliation3 \\ {\tt author3@ismir.edu}}
%  {Fourth author} {Affiliation4 \\ {\tt author4@ismir.edu}}

\begin{document}
%
\maketitle
%
\begin{abstract}
Musical performance combines a wide range of pitches, nuances,
and expressive techniques.
Audio-based classification of musical instruments thus requires to
build signal representations that are invariant to such transformations.
This article investigates the construction
of multi-stage architectures for instrument recognition,
given a limited amount of annotated training data.
We show that the main drawback of mel-frequency cepstral
coefficients (MFCC) resides in their lack of invariance with respect to
realistic pitch shifts, despite being designed to be invariant
to the frequency transposition of pure tones.
In turn, a deep convolutional network (ConvNet)
in the time-frequency domain is able to disentangle pitch from
timbral information, hence yielding better classification accuracy.
By recombining convolutional feature maps over the Shepard pitch spiral,
we further improve the learned representation
by introducing weight sharing strategies dedicated to
quasi-harmonic sounds with fixed spectral envelope,
which are archetypal of musical notes.
\end{abstract}

\section{Introduction}\label{sec:introduction}
Among the cognitive attributes of musical tones, pitch is distinguished
by a combination of three properties.
First, it is relative: ordering pitches from low to high gives rise to
intervals and melodic patterns.
Secondly, it is intensive: multiple pitches heard simultaneously produce
a chord, not a single unified tone -- contrary to loudness, which adds
up with the number of sources.
Thirdly, it does not depend on instrumentation: this makes possible
the transcription of polyphonic music under a single symbolic system
\cite{deCheveigne2005}.

Tuning auditory filters to a perceptual scale of pitches provides a
time-frequency representation of music signals that satisfies the first two of these properties.
It is thus a starting point for a wide range of MIR applications,
which can be separated in two categories: pitch-\emph{relative}
(\eg chord estimation \cite{Humphrey2012})
and pitch-\emph{invariant} (\eg instrument recognition \cite{Eronen2000}).
Both aim at disentangling pitch from timbral content as independent
factors of variability, a goal that is made possible by the third aforementioned property.
This is pursued by extracting mid-level features on top of the spectrogram,
be them engineered or learned from training data.
Both approaches have their limitations: a "bag-of-features" lacks flexibility
to represent fine-grain class boundaries, whereas a purely learned pipeline
often leads to uninterpretable overfitting, especially in MIR where the quantity
of thoroughly annotated data is relatively small.

In this article, we strive to integrate domain-specific knowledge about musical
pitch into a deep learning framework, in an effort towards bridging the gap
between feature engineering and feature learning.

Section 2 reviews the related work on feature learning for signal-based music
classification.
Section 3 demonstrates that pitch is the major factor of variability among musical
notes of a given instrument, if described by their mel-frequency cepstra.
Section 4 describes a typical deep learning architecture for spectrogram-based
classification, consisting of two convolutional layers and one densely connected layer.
Section 5 improves the previous architecture by splitting spectrograms into
octave-wide frequency bands, training specific convolutional layers over each band
in parallel, and gathering feature maps at a later stage.
Sections 6 discusses the effectiveness of the presented systems on a challenging
dataset for music instrument recognition.

\section{Related work}
Spurred by the growth of annotated datasets and the democratization of
high-performance computing, feature learning has enjoyed a renewed interest
in recent years within the MIR community.
Hand-crafted features, a mainstream method in MIR that requires human expert 
knowledge, tend to not generalize
well to large scale problems. This has been an important motivation for the design 
of efficient feature learning techniques,
both in supervised and unsupervised manners.
Whereas unsupervised learning (\eg $k$-means \cite{Stowell2014}, Gaussian
mixtures \cite{Joder2009}) is employed to fit the distribution of the data with
few parameters of relatively low abstraction
and high dimensionality, state-of-the-art supervised learning consists of a
composition of multiple nonlinear transformations, jointly optimized
to predict class labels, and whose behaviour gain in abstraction as depth
increases \cite{vandenOord2013}.

Among the supervised methods, convolutional representations provided the more
convincing results. The success of convolutional methods is probably motivated
by the translation invariant properties they implement by definition. A convolutional
representation, indeed, relies on the stationarity assumption for which the inputs are made
of small patches whose content is statistically independent
from their location within the signal, be it a one-dimensional waveform or
a two-dimensional spectrogram.
As a consequence, linear transformations of the data can be learned efficiently
by limiting their support to a small kernel which is convolved over
the whole input.
This method, known as weight sharing, decreases the number of parameters
of each feature map while increasing the amount of available training data.
In order to clarify the importance of learning in feature design, we will describe
some effective approaches on music and audio signals in general.

Newton and Smith \cite{Newton2012} have built a network to represent the temporal 
dynamics of isolated musical notes.
They created a neurally inspired tone descriptor using a model of the auditory system's 
response to sound onset, based on dynamic synapses and leaky integrate-and-fire neurons.  
Final classification is then performed using a specific time-domain neural network, 
called echo state network.

McFee \etal \cite{McFee2015-muda} have trained a deep convolutional network on
constant-Q representations of music instruments in order to illustrate the benefits of
artificial data augmentation. Using two convolutional layers followed by two fully-connected
layers, the authors successfully proved that the overfitting of the network is
greatly reduced by applying augmentation on data by means of pitch shift, 
time stretch and noise addition.

Li \etal \cite{Li2015} have also used a deep convolutional network but trained directly 
on raw audio waveforms
for polyphonic instrument recognition. Their aim was to prove that and end-to-end approach
(from the learning of features to the classification task) outperform more traditional
approaches.
The authors also remark that the filters learned in the first layer seem to be 
frequency selective and similar to an auditory scale filter bank.

Some other applications of deep convolutional networks include onset
detection \cite{Schluter2014}, transcription \cite{Sigtia2015},
genre classification \cite{Choi2015},
chord recognition \cite{Humphrey2012}, boundary detection \cite{Ullrich2014}, and
recommendation \cite{vandenOord2013}.

As a matter of facts, the most widely studied deep learning system for music information retrieval
consists of two convolutional layers and two densely connected layers,
with minor variations \cite{Dieleman2014, Humphrey2012, Kereliuk2015, Li2015,
McFee2015-muda, Schluter2014, Ullrich2014}.

There is little to none theoretical examination on this kind of networks and the reason
for which the discussed architecture is the most accepted for audio analysis is not 
totally clear.
Convolutional layers are aimed at learning factors of variability with different degree 
of abstraction.
The joint action of convolution filters and non linear operators seems to implement the so-called
\emph{contraction} and \emph{separation} properties of deep convolutional networks \cite{Mallat2015}: 
the former creates a non-linear reduction of the feature space to decrease the data variability 
along local symmetries. The latter, on the other hand, avoids that elements belonging 
to different classes clash together
as consequence of the contraction.

The deepness of the network is a mean of controlling the ratio between contraction and separation and, indirectly,
the abstraction level of the representation: we advance the hypothesis, in this context, that two layers 
of convolutions (with related non linearities) are the optimal trade-off for the complexity of the signals 
treated and for the size of the dataset used.

Many challenges, however, remain to be addressed. The process of gaining abstraction is somehow related,
in deep networks, to the change of scale by means of pooling operators. 
In musical signals, therefore, it would be interesting to find specific time-frequency patterns 
over longer time scales. Moreover, in the context of music instrument classification, it would be
interesting to build invariants to pitch and melody that preserve the separation among the instruments.

% The problem is made difficult by the fact that factors of variability are entangled.
% Deep convolutional networks have proven to disentangle factors of variability in computer
% vision, such as pose, color, and lighting conditions.

% the challenge is thus two-fold
% 1. gaining abstraction by integrating time-frequency patterns over longer time scales
% 2. building invariants to melody while remaining highly discriminative to the instrument

% On music instrument classification
% Ref to Joder et al
% Ref to Fuhrmann
% On feature learning
% Ref to Dieleman and Benjamin ICASSP 2014
% Ref to Humphrey, Bello, LeCun 2012
% Ref to Salamon and Bello

\section{How invariant is the Mel-frequency cepstrum ?}
The mel scale is a quasi-logarithmic function of acoustic frequency designed such that
perceptually similar pitch intervals appear equal in width over the full hearing range.
This section shows that engineering transposition-invariant features from the mel
scale does not suffice to build pitch invariants for complex sounds, thus motivating
further inquiry.

The time-frequency domain produced by a constant-Q filter bank tuned to the mel
scale is covariant with respect to pitch transposition of pure tones.
As a result, a chromatic scale played at constant speed would draw parallel,
diagonal lines, each of them corresponding to a different partial wave.
However, the physics of musical instruments constrain these partial waves to bear
a negligible energy if their frequencies are beyond the range of acoustic resonance.

As shown on Figure \ref{fig:chromatic-scale}, the constant-Q spectrogram of a
tuba chromatic scale exhibits a fixed, cutoff frequency at about 2500 Hz, which
delineates the support of its spectral envelope.
This elementary observation implies that realistic pitch changes cannot be modeled
by translating a rigid spectral template along the log-frequency axis.
The same property is verified for a wide class of instruments, especially brass and
woodwinds.
As a consequence, the construction of powerful invariants to musical pitch is not
amenable to delocalized operations on the mel-frequency spectrum, such as a
discrete cosine transform (DCT) which leads to the mel-frequency cepstral
coefficients (MFCC) classically used in music classification
\cite{Eronen2000, Joder2009}.

\begin{figure}[t]
    \begin{center}
        \setlength{\unitlength}{1cm}
        \begin{picture}(8.2,3)
        \put(0,-0.5){\includegraphics[width=8cm]{figs/chromatic_scale.png}}
        \end{picture}
    \end{center}
    \protect\caption{
    Constant-Q spectrogram of a chromatic scale played by a tuba.
    Although the harmonic partials shift progressively, the spectral envelope remains unchanged,
    as revealed by the presence of a fixed cutoff frequency.
    See text for details.
\label{fig:chromatic-scale}
}
\end{figure}

To validate the above claim, we have extracted the MFCC
of 1116 individual notes from the RWC dataset \cite{Goto2003},
as played by 6 instruments, with
32 pitches, 3 nuances,
and 2 interprets and manufacturers.
When more than 32 pitches were available (\eg piano), we selected
a contiguous subset of 32 pitches in the middle register.
Following a well-established rule \cite{Eronen2000, Joder2009},
the MFCC were defined the 12 lowest nonzero "quefrencies" among the
DCT coefficients extracted from a filter bank of 40 mel-frequency bands.
We then have computed the distribution of squared Euclidean distances
between musical notes in the 13-dimensional space of MFCC features.

Figure \ref{fig:mfcc-variances} summarizes our results.
We found that restricting the cluster to one nuance, one interpret, or one manufacturer
hardly reduces intra-class distances.
This suggests that MFCC are fairly successful in building invariant representations
to such factors of variability.
In contrast, the cluster corresponding to each instrument is shrinked if
decomposed into a mixture of same-pitch clusters, sometimes by an order of
magnitude.
In other words, most of the variance in an instrument cluster of mel-frequency
cepstra is due to pitch transposition.

Keeping less than 13 coefficients certainly improves invariance, yet at the cost of
inter-class discriminability, and vice versa.
This experiment shows that the mel-frequency cepstrum is perfectible in terms
of invariance-discriminability tradeoff, and that there remains a lot to be gained by
feature learning in this area.
\begin{figure}[t]
    \begin{center}
        \setlength{\unitlength}{1cm}
        \begin{picture}(8.5,8.7)
        \put(0.1,0){\includegraphics[width=8cm]{figs/mfcc_variances.png}}
        \end{picture}
    \end{center}
    \protect\caption{
Distributions of squared Euclidean distances among various MFCC clusters in the RWC dataset.
Whisker ends denote lower and upper deciles. See text for details.
\label{fig:mfcc-variances}
}
\end{figure}
\section{Deep convolutional networks}
A deep learning system for classification is built by stacking multiple layers of weakly nonlinear
transformations, whose parameters are optimized such that the top-level layer fits a training
set of labeled examples.
This section introduces a typical deep learning architecture for audio classification and describes
the functioning of each layer.

\begin{figure*}[t]
    \begin{center}
        \setlength{\unitlength}{1cm}
        \begin{picture}(17,5)
        \put(0,0){\includegraphics[width=17cm]{figs/architecture.png}}
        \end{picture}
    \end{center}
    \protect\caption{
Architecture of a convolutional network with full weight sharing. See text for details.
\label{fig:instrument-distribution}
}
\end{figure*}

The input of our system is a constant-Q wavelet scalogram, which is very comparable to a
mel-frequency spectrogram.
We used the implementation from the librosa package \cite{McFee2015-librosa} with $Q=12$
filters per octave, center frequencies ranging from 55 Hz to 14 kHz (8 octaves from A1 to A9),
and a hop size of 23 ms.
Furthermore, we applied nonlinear perceptual weighting of loudness in order to reduce the
dynamic range between the fundamental partial and its upper harmonics.
A 3-second sound excerpt $\boldsymbol{x}[t]$ is represented by a time-frequency matrix
$\boldsymbol{x_1}[t,k_1]$ of width $T=128$ samples and height $K_1=96$ frequency bands.

Each layer in a convolutional network typically consists in the composition of three operations:
two-dimensional convolutions, application of a pointwise nonlinearity, and local pooling.
A convolutional operator is defined as a family $\boldsymbol{W_2}[\tau,\kappa_1,k_2]$ of
$K_2$ two-dimensional filters, whose impulse repsonses are all constrained to have width
$\Delta t$ and height $\Delta k_1$. Element-wise biases $\boldsymbol{b_2}[k_2]$ are
added to the convolutions, resulting in the three-way tensor 
\begin{IEEEeqnarray}{rCl}
\IEEEeqnarraymulticol{3}{l}{
\boldsymbol{y_2}[t,k_1,k_2]} \nonumber \\
& = & \boldsymbol{b_2}[k_2] + 
\boldsymbol{W_2}[t, k_1, k_2] \overset{t,k_1}{\ast} \boldsymbol{x_1}[t, k_1]
\nonumber \\
& = &
\boldsymbol{b_2}[k_2] + 
\sum_{\substack{
0 \leq \tau < \Delta t \\
0 \leq \kappa_1 < \Delta k_1}}
\! \! \! \! \!
\boldsymbol{W_2}[\tau, \kappa_1, k_2]
\boldsymbol{x_1}[t-\tau, k_1-\kappa_1].
\IEEEeqnarraynumspace
\label{eq:convolution}
\end{IEEEeqnarray}
The pointwise nonlinearity we have chosen is the rectified linear unit (ReLU),
with a rectifying slope of $\alpha=0.3$ for negative inputs.
\begin{IEEEeqnarray}{l}
\boldsymbol{y_{2}^{+}}[t,k_{1},k_{2}]=\left\{ \! \! \! \begin{array}{r}
\alpha\boldsymbol{x_{2}}[t,k_{1},k_{2}] \;\;\; \mbox{if} \;\; \boldsymbol{x_{2}}[t,k_{1},k_{2}]<0\\
\boldsymbol{x_{2}}[t,k_{1},k_{2}] \;\;\; \mbox{if} \;\; \boldsymbol{x_{2}}[t,k_{1},k_{2}]>0
\end{array}\right. \!
\IEEEeqnarraynumspace
\label{eq:relu}
 \end{IEEEeqnarray}
The pooling step consists in retaining the maximal activation among neighboring units in the
time-frequency domain $(t, k_1)$ over non-overlapping rectangles of width $\Delta t$ and
height $\Delta k_1$.
\begin{equation}
\boldsymbol{x_2}[t,k_1,k_2] = \! \!
\max_{
\substack{
0 \leq \tau < \Delta t \\
0 \leq \kappa_1 < \Delta k_1}
} \! \!
\left\{
\boldsymbol{y_{2}^{+}}[t - \tau, k_1 - \kappa_1, k_2]
\right\}
\label{eq:pooling}
\end{equation}
The hidden units in $\boldsymbol{x_2}$ are in turn fed to a second layer of convolutions,
ReLU, and pooling.
Observe that the corresponding convolutional operator
$\boldsymbol{W_3}[\tau, \kappa_1, k_2, k_3]$ performs a linear combination of time-frequency
feature maps in $\boldsymbol{x_2}$ along the channel variable $k_2$.
\begin{IEEEeqnarray}{rCl}
\IEEEeqnarraymulticol{3}{l}{
\boldsymbol{y_3}[t,k_1,k_3]} \nonumber \\
& = &
\sum_{k_2}
\boldsymbol{b_3}[k_2, k_3]
+ \boldsymbol{W_3}[t, k_1, k_2, k_3]
\overset{t,k_1}{\ast}
\boldsymbol{x_2}[t,k_1,k_2].
\IEEEeqnarraynumspace
\end{IEEEeqnarray}
Tensors $\boldsymbol{y_3^{+}}$ and $\boldsymbol{x_3}$ are derived from $\boldsymbol{y_3}$
by ReLU and pooling, with formulae similar to Eqs. (\ref{eq:relu}) and (\ref{eq:pooling}).
The third layer consists of the linear projection of $\boldsymbol{x_3}$, viewed as a vector of
the flattened index $(t, k_1, k_3)$, over $K_4$ units:
\begin{IEEEeqnarray}{rCl}
\boldsymbol{y_4}[k_4] =
\boldsymbol{b_4}[k_4] +
\sum_{t,k_1,k_3}
\boldsymbol{W_4}[t, k_1, k_3, k_4]
\boldsymbol{x_3}[t, k_1, k_3]
\label{eq:densely-connected-layer}
\IEEEeqnarraynumspace
\end{IEEEeqnarray}
We apply a ReLU to $\boldsymbol{y_4}$, yielding
$\boldsymbol{x_4}[k_4] = \boldsymbol{y_4^{+}}[k_4]$.
Finally, we project $\boldsymbol{x_4}$ onto a layer of output units $\boldsymbol{y_5}$ that
should represent instrument activations:
$\boldsymbol{y_5}[k_5] = \sum_{k_4} \boldsymbol{W_5}[k_4, k_5] \boldsymbol{x_4}[k_4]$.
The final transformation is a softmax nonlinearity, which ensures that output coefficients are
non-negative and sum to one, hence can be fit to a probability distribution.
\begin{equation}
\boldsymbol{x_5}[k_5] =
\frac{\exp \boldsymbol{y_5}[k_5]}
{  \sum_{\kappa_5} \exp \boldsymbol{y_5}[\kappa_5] }
\end{equation}
The goal is to minimize the average loss $\mathscr{L}(\boldsymbol{x_5}, \mathcal{I})$
across all pairs $(\boldsymbol{x}, \mathcal{I})$ in the training set. This loss is defined as
the categorical cross-entropy over shuffled mini-batches of size 32 with uniform
class distribution, to which is added a weight decay term upon the last layer.
\begin{equation}
\mathscr{L}(\boldsymbol{x_5}, \mathcal{I}) =
- \sum_{k_5 \in \mathcal{I}} \log \boldsymbol{x_5}[k_5]
+ \lambda_5 \Vert W_5 \Vert_2
\end{equation}
Each training example is a 3-second spectrogram whose boundaries are selected at random
over non-silent regions of a song.
Each spectrogram within a batch was globally normalized
such that the whole batch had zero mean and unit variance.
The learning rate policy for each scalar weight in the network is Adam \cite{Kingma2015},
a state-of-the-art online optimizer for gradient-based learning.
Mini-batch training was stopped after the average training loss stopped
decreasing over one full epoch of size 8192.
The architecture was built using the Keras library \cite{Chollet2015},
and trained on a graphics processing unit within a few minutes.

\section{Improved weight sharing strategies}
Although a dataset of music signals is unquestionably stationary over the time
dimension -- at least at the scale of a few seconds -- it cannot be taken for granted
that all frequency bands of a mel-frequency spectrogram would have the same
local statistics \cite{Humphrey2013}.
In this section, we introduce two alternative architectures to address
nonstationarity of music on the mel-frequency axis,
while still leveraging the efficiency of convolutional representations in the
time-frequency domain.
 
Some objections the stationarity assumption among local neighborhoods in frequency
are the following:
the spectral envelope of musical instruments remains fixed in spite of
pitch changes (see Section 2) ;
partials of a harmonic comb get closer to each other in high frequencies on a mel scale ;
due to the Heisenberg principle, the temporal resolution of auditory filters is lessened at
lower frequencies.

\subsection{One-dimensional convolutions at high frequencies}
Facing nonstationary constant-Q spectra,
the most conservative workaround is to increase the height $\Delta \kappa_1$ of each
convolutional kernel up to the total number of bins $K_1$ in the spectrogram.
As a result, $\boldsymbol{W_1}$ and $\boldsymbol{W_2}$ are no longer transposed
over adjacent frequency bands, since convolutions are merely performed over
the time variable.
The definition of $\boldsymbol{y_2}[t, k_1, k_2]$ rewrite as
\begin{IEEEeqnarray}{rCl}
\IEEEeqnarraymulticol{3}{l}{
\boldsymbol{y_2}[t,k_1,k_2]} \nonumber \\
& = & \boldsymbol{b_2}[k_2] + 
\boldsymbol{W_2}[t, k_1, k_2] \overset{t}{\ast} \boldsymbol{x_1}[t, k_1]
\nonumber \\
& = &
\boldsymbol{b_2}[k_2] + 
\sum_{\substack{0 \leq \tau < \Delta t}}
\! \! \! \! \!
\boldsymbol{W_2}[\tau, k_1, k_2]
\boldsymbol{x_1}[t-\tau, k_1],
\IEEEeqnarraynumspace
\label{eq:convolution}
\end{IEEEeqnarray}
and similarly for $\boldsymbol{y_3}[t, k_1, k_3]$.
While this approach is theoretically capable of encoding pitch invariants, it is
prone to early specialization of low-level features, thus
not fully taking advantage of the network depth.

However, the situation is improved if the one-dimensional
kernels are restricted to the highest frequencies of the constant-Q spectrum.
It should be observed that, around the $n^{\textrm{th}}$ partial of a quasi-harmonic sound,
the distance in log-frequency between neighboring partials decays like $1/n$,
and the unevenness between those distances decays like $1/n^2$.
Consequently, at the topmost octaves of the constant-Q spectrum, 
where $n$ is equal or greater than $Q$, the partials appear close to each other and almost
evenly spaced.
Furthermore, due to the logarithmic compression of loudness, the polynomial decay
of the spectral envelope is linearized: thus, at high frequencies, transposed pitches
have similar spectra up to some additive bias.
The combination of these two phenomena implies that the correlation between
constant-Q spectra of different pitches is greater towards high frequencies, and that
the learning of polyvalent feature maps becomes tractable.

\subsection{Convolutions on the pitch spiral at low frequencies}
The weight sharing strategy presented above exploits the facts that,
at high frequencies, quasi-harmonic partials are numerous, and that the
amount of energy within a frequency band is independent of pitch.
At low frequencies, we make the exact opposite assumptions: we claim
that the harmonic comb is sparse and covariant with respect to pitch shift.
Observe that, for any two distinct partials taken at random between $1$ and $n$,
the probability that they are in octave relation is slightly above $1/n$.

Leveraging the fact that power-of-two harmonics are one octave apart, we roll
up the log-frequency into a Shepard pitch spiral, such that octave intervals
correspond to full turns.

\begin{IEEEeqnarray}{rCl}
\boldsymbol{y_2}[t,k_1,k_2]
= & &
\! \! \! \! \! \! \! \! \! \! \! \! \! \! \! \! \! \! \! \!
\boldsymbol{b_2}[k_2]  \nonumber \\
& +
\! \sum_{\tau, \kappa_1, j_1} \! &
\boldsymbol{W_2}[\tau, \kappa_1, j_1, k_2] \nonumber \\
& &\times
\boldsymbol{x_1}[t - \tau, k_1 - \kappa_1 - Q j_1]
\IEEEeqnarraynumspace
\end{IEEEeqnarray}

The linear combinations of frequency bands that are one octave apart,
as proposed here,
bears a resemblance with engineered features for music instrument
recognition \cite{Peeters2004}, such as tristimulus, 
empirical inharmonicity, harmonic spectral deviation,
odd-to-even harmonic energy ratio, as well as
octave band signal intensities (OBSI) \cite{Joder2009}.

In summary, the classical two-dimensional convolutions make a stationarity assumption
among frequency neighborhoods. This approach gives a coarse approximation
of the spectral envelope.
Resorting to one-dimensional convolutions allows to disregard nonstationarity,
but does not yield a pitch-invariant representation per se:
thus, we only apply them at the topmost frequencies, \ie where the
invariance-to-stationarity ratio in the data is already favorable.
Conversely, two-dimensional convolutions on the pitch spiral addresses
the invariant representation of sparse, transposition-covariant spectra:
they are best suited to the lowest frequencies, \ie where partials
are further apart and pitch changes can be approximated by log-frequency
translations.

%Whereas the resort to one-dimensional convolutions
%leverages the fact that quasi-harmonic partials are more numerous towards
%high frequencies, we make the exact opposite assumption at low frequencies.

%To address this issue, Abdel-Hamid \etal \cite{Abdel-Hamid2014} have developed
%limited weight sharing in the first convolutional layer, and shown that it improves
%the performance of an end-to-end speech recognition system.
%Limited weight sharing consists in splitting the spectrogram into subbands, whose bandwidth
%is of the order of one octave.
%Subbands are then fed in parallel to a layer of convolutions, nonlinearity, and pooling.
The outputs of each transformed subband are finally aggregated into an unstructured
vector.

\section{Applications}\label{sec:single-instrument}
In order to train the proposed algorithms, we used MedleyDB v1.1. \cite{Bittner2014}, a
dataset of 122 multitracks annotated with instrument activations as well as melodic $f_0$
curves when present. 
We extracted the monophonic stems corresponding to a selection of eight pitched
instruments (see Table \ref{table:single-label-durations}).
Stems with leaking instruments in the background were discarded.

\begin{table}
	\begin{center}
	\begin{tabular}{|c|cc|cc|}
		\hline
		& minutes & tracks & minutes & tracks \\
		\hline
		clarinet & 10 & 7 & 13 & 18 \\
		dist. guitar & 15 & 14 & 17 & 11 \\
		female singer & 10 & 11 & 19 & 12 \\
		flute & 7 & 5 & 53 & 29 \\
		piano & 58 & 28 & 44 & 15 \\
		tenor sax. & 3 & 3 & 6 & 5 \\
		trumpet & 4 & 6 & 7 & 27 \\
		violin & 51 & 14 & 49 & 22 \\
		\hline
		total & 158 & 88 & 208 & 139 \\
		\hline
	\end{tabular}
	\end{center}
	\caption{\label{table:single-label-durations}}
\end{table}

The evaluation set consists of 126 recordings of solo music collected by
Joder \etal \cite{Joder2009}, supplemented with
23 stems of electric guitar and female voice from MedleyDB.
In doing so, guitarists and vocalists were thoroughly put either in the training set or the test set,
to prevent any artist bias.
We discarded recordings with extended instrumental techniques, since they are
extremely rare in MedleyDB.

Constant-Q spectrograms from the evaluation set were split into half-overlapping,
3-second excerpts.
The predicted probability distributions were computed for every excerpt in a track,
and then aggregated by geometric mean to provide a decision at the scale of
the entire audio file.

In order to compare the results against shallow classifiers, we also extracted a typical
"bag-of-features" over half-overlapping, 3-second excerpts in the training set.
These features consist of the temporal averages and standard
deviations of spectral shape descriptors, \ie centroid, bandwidth, skewness,
and rolloff, as well a zero-crossing rate ;
supplemented with the temporal averages of MFCC and its first and second derivatives.
We trained a random forest of 100 decision trees on the resulting feature vector
of dimension 
\section{Conclusions}
Understanding the influence of pitch in audio streams is paramount to the design of
an efficient system for automated classification, tagging, and similarity retrieval in music. 
We have presented a data-driven, supervised method to address pitch invariance
while preserving good timbral discriminability.


Future work will be devoted to integrating the proposed scheme with other advances
in deep learning for music informatics, such as data augmentation \cite{McFee2015-muda},
multiscale representations \cite{Hamel2012, Anden2015},
and adversarial training \cite{Kereliuk2015}.
% For bibtex users:
\bibliography{ISMIR2015template}

\end{document}
